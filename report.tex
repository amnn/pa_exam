\documentclass{tufte-handout}

\title{Program Analysis 2016}
\author{Candidate No. 683444}

\usepackage[ruled,vlined]{algorithm2e}
\usepackage{amsmath}
\usepackage{amssymb}
\usepackage{amsthm}
\usepackage{bm}
\usepackage{bussproofs}
\usepackage{calc}
\usepackage{enumerate}
\usepackage{mathtools}
\usepackage{relsize}
\usepackage{stmaryrd}
\usepackage{tikz}
\usepackage{wasysym}

\usetikzlibrary{cd}

%%% Title Formatting

\titleformat*{\section}{\normalfont\Large\scshape}
\titleformat*{\subsection}{\normalfont\large\scshape}
\newcommand{\sectionbreak}{\clearpage}

%%% Custom Commands

\newcommand{\Lemma}{\textbf{lemma}}
\newcommand{\Thm}{\textbf{thm}}
\newcommand{\Def}{\textbf{def}}
\newcommand{\Contra}{\textbf{contra}}
\newcommand{\Assoc}{\textbf{assoc}}
\newcommand{\Trans}{\textbf{trans}}
\newcommand{\Refl}{\textbf{refl}}
\newcommand{\Hyp}{\textbf{hyp}}
\newcommand{\Ass}{\textbf{assume}}
\newcommand{\Intro}{\textbf{intro}}
\newcommand{\Elim}{\textbf{elim}}
\newcommand{\Let}{\textbf{let}}
\newcommand{\Epic}{\textbf{epic}}
\newcommand{\Monic}{\textbf{monic}}
\newcommand{\compose}{\circ}
\newcommand{\epic}{\twoheadrightarrow}
\newcommand{\cat}[1]{\mathcal{#1}}
\newcommand{\opcat}[1]{\mathcal{#1}^{\text{op}}}
\renewcommand{\hom}[3]{\cat{#1}(#2, #3)}
\newcommand{\Exp}{\Rightarrow}
\newcommand{\sembrack}[1]{\llbracket #1 \rrbracket}

\newcommand{\step}[1][\phantom{=}]{\item[{\makebox[{\widthof{$\Leftrightarrow$}}][l]{$#1$}}]}
\newcommand{\subp}[1]{\item[$#1$]}
\newcommand{\iffs}{\Leftrightarrow}
\newcommand{\imps}{\Rightarrow}
\newcommand{\contras}{\divideontimes}

\def\mathnote#1{%
  \tag*{\rlap{\hspace\marginparsep\smash{\parbox[t]{\marginparwidth}{%
  \footnotesize#1}}}}
}

\DeclarePairedDelimiter\abs{\lvert}{\rvert}
\DeclareMathOperator{\id}{id}
\DeclareMathOperator{\FV}{FV}
\DeclareMathOperator{\App}{App}
\DeclareMathOperator{\Widen}{\nabla}
\DeclareMathOperator{\Narrow}{\triangle}
\DeclareMathOperator{\Infty}{\mathit{infty}}

%%% Theorem styles

\theoremstyle{definition}
\newtheorem{definition}{Definition}
\numberwithin{definition}{section}

\theoremstyle{plain}
\newtheorem{prop}{Proposition}
\numberwithin{prop}{section}

\theoremstyle{plain}
\newtheorem{lemma}{Lemma}
\numberwithin{lemma}{section}

\theoremstyle{plain}
\newtheorem{corollary}{Corollary}
\numberwithin{corollary}{section}

%%% Initialise Counters

\setcounter{section}{1}
\maxdeadcycles=1000

%%% Proof Trees
\EnableBpAbbreviations

%%% Content
\begin{document}
\maketitle

\section{Question 1}\label{sec:q-1}
\subsection{Part (a)}\label{sec:q-1-a}

Suppose we have two variables, $\mathit{Var}= \{x_1, x_2\}$. Consider

\begin{enumerate}[-]\itemsep0em
\item $(m_n)_{n\geq0}\in\mathit{Zon}^\omega$, a sequence, and,
\item $(w_n)_{n\geq0}$, its widenings according to $\Widen_1$.
\end{enumerate}

\noindent
Defined respectively according to:

\begin{align*}
  m_{2i} & =
  \begin{bmatrix}
    0 & 1 & \infty \\
    1 & 0 & 1 \\
    1 & 1 & 0
  \end{bmatrix}
  && \text{ and } &
  m_{2i+1} & =
  \begin{bmatrix}
    0 & \infty & 1 \\
    1 & 0 & 1 \\
    1 & 1 & 0
  \end{bmatrix}
  && \text{ for } i\geq0\\
  w_0 & = m_0
  && \text{ and } &
  w_{i+1} & = w_i^\ast\Widen m_{i+1}
  && \text{ for } i\geq0
\end{align*}

\begin{prop}\label{prop:widen-1}
  \begin{align*}
    w_i =
    \begin{bmatrix}
      0 & i + 1 & \infty\\
      1 & 0 & 1\\
      1 & 1 & 0
    \end{bmatrix}
    \text{ if } i \text{ is even and }
    \begin{bmatrix}
      0 & \infty & i + 1 \\
      1 & 0 & 1\\
      1 & 1 & 0
    \end{bmatrix}
    \text{ if } i \text{ is odd}
  \end{align*}

  \noindent
  Proof by induction on $i$.

  \begin{proof}[Base Case, $i = 0$]
    \begin{align*}
      w_0 = m_0 =
      \begin{bmatrix}
        0 & 1 & \infty \\
        1 & 0 & 1 \\
        1 & 1 & 0
      \end{bmatrix}
      \tag*{\qedhere}
    \end{align*}
  \end{proof}

  \begin{proof}[Inductive step $i = 2k$, $k > 0$]
    \begin{align*}
      w_{2k} &= w_{2k-1}^\ast\Widen m_{2k}
      \\ &= w_{2(k-1)+1}^\ast\Widen m_{2k}
      \\ &=
      \begin{bmatrix}
        0 & \infty & 2k \\
        1 & 0 & 1 \\
        1 & 1 & 0
      \end{bmatrix}^\ast
      \Widen
      \begin{bmatrix}
        0 & 1 & \infty \\
        1 & 0 & 1 \\
        1 & 1 & 0
      \end{bmatrix}
      \mathnote{Induction Hypothesis and definition of $m$ (at even indices).}
      \\ &=
      \begin{bmatrix}
        0 & 2k+1 & 2k \\
        1 & 0 & 1 \\
        1 & 1 & 0
      \end{bmatrix}^\ast
      \Widen
      \begin{bmatrix}
        0 & 1 & \infty \\
        1 & 0 & 1 \\
        1 & 1 & 0
      \end{bmatrix}
      \mathnote{Definition of closure}
      \\ &=
      \begin{bmatrix}
        0 & 2k+1 & \infty \\
        1 & 0 & 1 \\
        1 & 1 & 0
      \end{bmatrix}
      \mathnote{Definition of widening\qedhere}
    \end{align*}
  \end{proof}

  \begin{proof}[Inductive step $i = 2k+1$, $k \geq 0$]
    \begin{align*}
      w_{2k+1} &= w_{2k}^\ast\Widen m_{2k+1}
      \\ &=
      \begin{bmatrix}
        0 & 2k+1 & \infty \\
        1 & 0 & 1 \\
        1 & 1 & 0
      \end{bmatrix}^\ast
      \Widen
      \begin{bmatrix}
        0 & 1 & \infty \\
        1 & 0 & 1 \\
        1 & 1 & 0
      \end{bmatrix}
      \mathnote{Induction Hypothesis and definition of $m$ (at odd indices).}
      \\ &=
      \begin{bmatrix}
        0 & 2k+1 & 2k+2 \\
        1 & 0 & 1 \\
        1 & 1 & 0
      \end{bmatrix}^\ast
      \Widen
      \begin{bmatrix}
        0 & 1 & \infty \\
        1 & 0 & 1 \\
        1 & 1 & 0
      \end{bmatrix}
      \mathnote{Definition of closure}
      \\ &=
      \begin{bmatrix}
        0 & \infty & 2k+2 \\
        1 & 0 & 1 \\
        1 & 1 & 0
      \end{bmatrix}
      \mathnote{Definition of widening\qedhere}
    \end{align*}
  \end{proof}
\end{prop}

\noindent
It follows as a Corollary to Proposition~\ref{prop:widen-1}, that $\forall k\in\mathbb{N}.~w_k\neq w_{k+1}$.

\subsection{Part (b.i)}\label{sec:q-1-b-i}
\begin{prop}\label{prop:narrow-lb}
  $\forall m,m^\prime\in\mathit{Zon}.~\gamma_z(m)\cap\gamma_z(m^\prime)~\subseteq~\gamma_z(m\Narrow_1 m^\prime)~\subseteq~\gamma_z(m)$
  \begin{proof}
    Let $m,m^\prime\in\mathit{Zon}$.
    \begin{itemize}\itemsep1em
      \step $m^\ast\sqcap m~\sqsubseteq~m^\ast\Narrow m^\prime~\sqsubseteq~m^\ast$
      \marginnote{$\Narrow$ is a narrowing operator}
      \step[\iffs] $\alpha_z(\gamma_z(m))\sqcap m^\prime~\sqsubseteq~m^\ast\Narrow m^\prime~\sqsubseteq~\alpha_z(\gamma_z(m))$
      \marginnote{Definition of closure}
      \step[\imps] $\gamma_z(\alpha_z(\gamma_z(m))\sqcap m^\prime)~\subseteq~\gamma_z(m^\ast\Narrow m^\prime)~\subseteq~\gamma_z(\alpha_z(\gamma_z(m)))$
      \marginnote{$\gamma_z$ is monotonic}
      \step[\imps] $\gamma_z(\alpha_z(\gamma_z(m)))\cap\gamma_z(m^\prime)~\subseteq~\gamma_z(m^\ast\Narrow m^\prime)~\subseteq~\gamma_z(\alpha_z(\gamma_z(m)))$
      \marginnote{$\gamma_z$ preserves meets}
      \step[\iffs] $\gamma_z(m)\cap\gamma_z(m^\prime)~\subseteq~\gamma_z(m^\ast\Narrow m^\prime)~\subseteq~\gamma_z(m)$
      \marginnote{$\gamma_z = \gamma_z\compose\alpha_z\compose\gamma_z$ from problem sheet 1, q2b}
      \step[\iffs] $\gamma_z(m)\cap\gamma_z(m^\prime)~\subseteq~\gamma_z(m\Narrow_1m^\prime)~\subseteq~\gamma_z(m)$
      \marginnote{Definition of $\Narrow_1$\qedhere}
    \end{itemize}
  \end{proof}
\end{prop}

\subsection{Part (b.ii)}\label{sec:q-1-b-ii}

The main proof of this result relies on the definition of a function $\Infty : \mathit{Zon}\to\mathbb{N}$,

\begin{align*}
  \\ \Infty(m) & = \sum_{i,j = 1}^{n+1}\mathbb{1}[m_{ij} = \infty]
  \\ \Infty(\bot) & = 0
\end{align*}

And some lemmas relating to how it interacts with narrowing and closure, which may be found after the proof itself.

\begin{prop}
  For any $(m_k)_{k\geq0}\in\mathit{Zon}^\omega$, let $(v_k)_{k\geq0}$ be the narrowed sequence according to $\Narrow_1$, such that:
  \begin{align*}
    v_0 & = m_0
    \\ v_{k+1} & = v_k^\ast\Narrow m_{k+1} && \text{for } k\geq0
  \end{align*}
  Then, there exists some $k_0\in\mathbb{N}$ such that $v_{k_0} = v_{k_0+1}$.

  \begin{proof}
    Consider any contiguous triple of elements in $(v_k)_{k\geq0}$; $v_i,v_{i+1},v_{i+2}$, such that:
    \begin{align*}
      v_i\neq(v_{i+1} = v_i^\ast\Narrow m_{i+1})\neq(v_{i+2} = v_{i+1}^\ast\Narrow m_{i+2})
    \end{align*}
    Observe that:
    \begin{align*}
      \Infty(v_{i+2})
      \overset{\text{(1)}}\leq\Infty(v_{i+1}^\ast)
      \overset{\text{(2)}}\leq\Infty(v_{i+1})
      \overset{\text{(1)}}\leq\Infty(v_i^\ast)
      \overset{\text{(2)}}\leq\Infty(v_i)
    \end{align*}
    Where
    \begin{enumerate}[(1)]
      \item follows by Corollary~\ref{cor:infty-narrow}
      \item follows by Lemma~\ref{lemma:infty-closure}
    \end{enumerate}

    Then we may show that for any such triple, at least one of the inequalities is strict (we proceed by cases)

    \begin{proof}[Case 1]
      Suppose $\Infty(v_{i+1}) < \Infty(v_i)$
      \begin{itemize}
        \step[\imps] Our conclusion trivially follows from the assumption.\qedhere
      \end{itemize}
    \end{proof}
    \begin{proof}[Case 2]
      Suppose $\Infty(v_{i+1}) = \Infty(v_i)$.
      \begin{itemize}
        \step[\iffs] $\Infty(v_i^\ast\Narrow m_{i+1}) = \Infty(v_i)$
        \marginnote{Definition of $v_{i+1}$}
        \step[\iffs] $\Infty(v_i^\ast\Narrow m_{i+1}) \geq \Infty(v_i^\ast)$
        \marginnote{Lemma~\ref{lemma:infty-closure}}
        \step[\wedge] $\Infty(v_i^\ast\Narrow m_{i+1})\leq\Infty(v_i^\ast)$
        \marginnote{Corollary~\ref{cor:infty-narrow}}
        \step[\imps] $\Infty(v_i^\ast\Narrow m_{i+1}) = \Infty(v_i^\ast)$
        \marginnote{Antisymmetry of $\leq$ }
        \step[\imps] $v_i^\ast\Narrow m_{i+1} = v_i^\ast$
        \marginnote{Lemma~\ref{lemma:infty-narrow} (negated)}
        \step[\iffs] $v_{i+1} = v_i^\ast$
      \end{itemize}
      Taking this result and substituting into the definition of $v_{i+2}$:
      \begin{align*}
        v_{i+2} & = v_{i+1}^\ast\Narrow m_{i+2}
        \\ & = v_i^{\ast\ast}\Narrow m_{i+2}
        \\ & = v_i^\ast\Narrow m_{i+2}
        \mathnote{Lemma~\ref{lemma:closure-idemp}}
        \\ & \neq v_{i+1}
        \mathnote{By assumption}
        \\ & = v_i^\ast
      \end{align*}
      \begin{itemize}
        \step[\imps] $\Infty(v_i^\ast\Narrow m_{i+2}) < \Infty(v_i^\ast)$
        \marginnote{Lemma~\ref{lemma:infty-narrow}}
        \step[\iffs] $\Infty(v_{i+2}) < \Infty(v_{i+1})$\qedhere
        \marginnote{Definitions of $v_{i+2}$ and $v_{i+1}$}
      \end{itemize}
    \end{proof}

    Given that
    \begin{enumerate}[(1)]
      \item $\forall m\in\mathit{Zon}.~0\leq\Infty(m)\leq(n+1)^2$
      \item By the above reasoning, at least every other element in any subsequence of $(v_k)_{k\geq0}$ has strictly fewer $\infty$'s than its predecessor.
    \end{enumerate}
    \begin{itemize}
      \step There are $O(n^2)$ contiguous elements in a non-terminating subsequence of $(v_k)_{k\geq0}$.
      \step[\imps] $(v_k)_{k\geq0}$ terminates in $O(n^2)$ steps.\qedhere
    \end{itemize}
  \end{proof}
\end{prop}

\begin{lemma}\label{lemma:infty-narrow}
  $\forall m,m^\prime\in\mathit{Zon}.~\Infty(m\Narrow m^\prime)<\Infty(m)\iff m\neq m\Narrow m^\prime$
  \begin{proof}[Proof $\Rightarrow$]
    This direction is trivial. If the number of infinities in two matrices are different, they cannot be equal.\qedhere
  \end{proof}

  \begin{proof}[Proof $\Leftarrow$]
    Suppose $m\neq m\Narrow m^\prime$. By the definition of $\Narrow$, we can make the following observations:
    \begin{enumerate}[(i)]
      \item $\forall1\leq i,j\leq n+1.~(m\Narrow m^\prime)_{ij}=\infty\implies m_{ij} = \infty$\\
        As $(\text{---}\Narrow m^\prime)$ does not introduce new $\infty$ elements.
      \item $\exists1\leq i,j\leq n+1.~m_{ij}=\infty\wedge(m\Narrow m^\prime)_{ij}\neq\infty$\\
        As $(\text{---}\Narrow m^\prime)$ changed $m$.
    \end{enumerate}
    From these observations, it follows that $\Infty(m\Narrow m^\prime)<\Infty(m)$.\qedhere
  \end{proof}
\end{lemma}

\begin{corollary}\label{cor:infty-narrow}
  $\forall m^\prime\in\mathit{Zon}.~\Infty\compose(\text{---}\Narrow m^\prime)\leq\Infty$
  \begin{proof}
    Follows from Lemma~\ref{lemma:infty-narrow}\qedhere.
  \end{proof}
\end{corollary}

\begin{lemma}\label{lemma:infty-closure}
  $\Infty\compose\alpha_z\compose\gamma_z\leq\Infty$
  \begin{proof}[Proof by contradiction]
    Suppose for a contradiction ${\exists m\in\mathit{Zon}.~\Infty(\alpha_z(\gamma_z(m))) > \Infty(m)}$
  \end{proof}
\end{lemma}

\begin{lemma}\label{lemma:closure-idemp}
  Closure is idempotent, that is to say
  ${(\alpha\compose\gamma)\compose(\alpha\compose\gamma) = \alpha\compose\gamma}$
  \begin{proof}~\\
    \begin{itemize}
      \step $(\alpha\compose\gamma)\compose(\alpha\compose\gamma)$
      \step[=] $\alpha\compose(\gamma\compose\alpha\compose\gamma)$
      \marginnote{Associativity of $\compose$}
      \step[=] $\alpha\compose\gamma$\qedhere
      \marginnote{$\gamma = \gamma\compose\alpha\compose\gamma$ from sheet 1 q2b.}
    \end{itemize}
  \end{proof}
\end{lemma}

\subsection{Part (c)}\label{sec:q-1-c}


\section{Question 2}\label{sec:q-2}
\subsection{Part (a)}\label{sec:q-2-a}
Define $\alpha : P(C)\to P(D)$ and $\gamma : P(D)\to P(C)$ by:
\begin{align*}
  \alpha(C_0) & = \{\beta(c) : c\in C_0\} \\
  \gamma(D_0) & = \{c\in C : \beta(c)\in D_0\}
\end{align*}

\begin{prop}
  $\alpha : P(C)\rightleftarrows P(D) : \gamma$ forms a Galois Connection\\
  It suffices to show that
  \begin{enumerate}[(i)]
    \item $\alpha$ is monotone
    \item $\gamma$ is monotone
    \item $\forall C_0\subseteq C,D_0\subseteq D.~\alpha(C_0)\subseteq D\iff C_0\subseteq\gamma(D_0)$
  \end{enumerate}
  \begin{proof}[Proof (i)]
    Let $C_0\subseteq C_0^\prime\subseteq C$
    \begin{itemize}
      \step Suppose $d\in\alpha(C_0)$
      \step[\imps] $d = \beta(c)$ for some $c\in C_0$
      \marginnote{Definition of $\alpha$}
      \step[\imps] $d = \beta(c)$ for some $c\in C_0^\prime$
      \marginnote{$C_0\subseteq C_0^\prime$}
      \step[\imps] $d\in\alpha(C_0^\prime)$
      \marginnote{Definition of $\alpha$}
      \step[\imps] $\alpha(C_0)\subseteq\alpha(C_0^\prime)$\qedhere
    \end{itemize}
  \end{proof}
  \begin{proof}[Proof (ii)]
    Let $D_0\subseteq D_0^\prime\subseteq D$
    \begin{itemize}
      \step Suppose $c\in\gamma(D_0)$
      \step[\imps] $\beta(c) \in D_0$
      \marginnote{Definition of $\gamma$}
      \step[\imps] $\beta(c) \in D_0^\prime$
      \marginnote{$D_0\subseteq D_0^\prime$}
      \step[\imps] $c\in\gamma(D_0^\prime)$
      \marginnote{Definition of $\gamma$}
      \step[\imps] $\gamma(D_0)\subseteq\gamma(D_0^\prime)$\qedhere
    \end{itemize}
  \end{proof}
  \begin{proof}[Proof (iii) $\Rightarrow$]
    Let $C_0\subseteq C, D_0\subseteq D$ such that $\alpha(C_0)\subseteq D_0$
    \begin{itemize}
      \step[\imps] $\{\beta(c) : c\in C_0\}\subseteq D_0$
      \marginnote{Definition of $\alpha$}
      \step[\imps] $\forall c\in C_0.~\beta(c)\in D_0$\hfill($\star$)
      \step let $c\in C_0$
      \step[\imps] $\beta(c)\in D_0$
      \marginnote{by $\star$}
      \step[\imps] $c\in\gamma(D_0)$
      \marginnote{Definition of $\gamma$}
      \step[\imps]$C_0\subseteq\gamma(D_0)$\qedhere
    \end{itemize}
  \end{proof}
  \begin{proof}[Proof (iii) $\Leftarrow$]
    Let $C_0\subseteq C, D_0\subseteq D$ such that $C_0\subseteq\gamma(D_0)$
    \begin{itemize}
      \step[\imps] $C_0\subseteq\{c\in C : \beta(c)\in D_0\}$
      \marginnote{Definition of $\gamma$}
      \step[\imps] $\forall c\in C_0.~\beta(c)\in D_0$\hfill($\star$)
      \step let $d\in \alpha(C_0)$
      \step[\imps] $d = \beta(c)$ for some $c\in C_0$
      \marginnote{Definition of $\alpha$}
      \step[\imps] $d\in D_0$
      \marginnote{by $\star$}
      \step[\imps]$\alpha(C_0)\subseteq D_0$\qedhere
    \end{itemize}
  \end{proof}
\end{prop}

\subsection{Part (b)}\label{sec:q-2-b}
Define $G : P(D)\to P(D)$
\begin{align*}
  G(D_0) = \{d_I\}\cup\{g(d) : d\in D_0\}
\end{align*}
\begin{prop}
  $\beta$ surjective $\implies G = \alpha\compose F\compose\gamma$
  \begin{proof}
    let $D_0\subseteq D$
    \begin{itemize}
      \step $\alpha(F(\gamma(D_0)))$
      \step[=] $\{\beta(c):c\in F(\gamma(D_0))\}$
      \marginnote{Definition of $\alpha$}
      \step[=] $\{\beta(c):c\in \{c_I\}\cup\{f(c):c\in\gamma(D_0)\}\}$
      \marginnote{Definition of $F$}
      \step[=] $\{\beta(c):c\in \{c_I\}\cup\{f(c):c\in\{c\in C:\beta(c)\in D_0\}\}\}$
      \marginnote{Definition of $\gamma$}
      \step[=] $\{\beta(c_I)\}\cup\{\beta(f(c)) : c\in C,\beta(c)\in D_0\}$
      \step[=] $\{d_I\}\cup\{g(\beta(c)) : c\in C,\beta(c)\in D_0\}$
      \marginnote{$\beta(c_I) = d_I$ and $\beta\compose f = g\compose\beta$}
      \step[=] $\{d_I\}\cup\{g(d):d\in D_0\}$
      \marginnote{$\beta$ surjective, by assumption}
      \step[=] $G(D_0)$
      \step[\imps] $G = \alpha\compose F\compose\gamma$\qedhere
    \end{itemize}
  \end{proof}
\end{prop}

\subsection{Part (c)}\label{sec:q-2-c}

Let $(D_k)_{k\geq0}$ be the sequence whose fixed-point is computed by our algorithm(s), defined by:
\begin{align*}
  D_0 & =\varnothing\\
  D_{k+1} & = G(D_k) && \text{for }k\geq0
\end{align*}

In order to calculate the worst-case time complexity, we note that Corollary~\ref{cor:g-abs} (given below) implies that if $(D_k)_{k\geq0}$ has not terminated after $\abs{D} + 1$ steps, $\abs{D_{\abs{D}+1}} = \abs{D} + 1$, which is a contradiction as $D$ is finite and $D_{\abs{D}+1}\subseteq D\implies \abs{D_{\abs{D}+1}}\leq \abs{D}$.\\[1em]

\noindent
$\implies (D_k)_{k\geq0}$ terminates in at most $\abs{D}+1$ steps.\\[1em]

As given in the question, every iteration of the algorithm takes, in the worst-case $O(\abs{D})$ time to run, so overall, this algorithm has a worst-case time complexity of $O(\abs{D}^2)$.\\[1em]

Taking inspiration from Corollary~\ref{cor:g-seq}, we may formulate an alternative algorithm as follows:

\begin{algorithm}
  \caption{An improved algorithm for calculating the fixed-point of $G$.}\label{alg:fix-g}
  \Begin{
      $D\gets\varnothing$\;
      $d\gets d_I$\;
      \While {$d\notin D$} {
        $D\gets D\cup\{d\}$\;
        $d\gets g(d)$
      }
      \KwRet $D$\;
    }
\end{algorithm}

As before, the outer-loop will take $\abs{D}+1$ iterations to terminate in the worst-case, but, the loop body now involves only a membership query, and an insertion, each taking $O(\log\abs{D})$ time in the worst-case, followed by the application of $g$ which we assume to be of unit cost.\\[1em]

\noindent
$\implies$ The worst-case time complexity of Algorithm~\ref{alg:fix-g} is $O(\abs{D}\log\abs{D})$.

\begin{prop}[Maintenance]
  At the beginning of iteration $i$, ${d = g^i(d_I)}$ and ${D = D_i}$.\\[1em]

  \noindent
  Proof by Induction
  \begin{proof}[Base Case $i = 0$]
    \begin{align*}
      D &= \varnothing = D_0\\
      d &= d_I = g^0(d_I)\tag*{\qedhere}
    \end{align*}
  \end{proof}
  \begin{proof}[Inductive Step $i = k + 1$]~\\
    \begin{itemize}
      \step At the beginning of iteration $k$, $D = D_k$, $d = g^k(d_I)$
      \marginnote{Induction Hypothesis}
      \step $d\notin D$ At the beginning of iteration $k$
      \marginnote{By assumption, we have reached the beginning of the $k+1^{\text{th}}$ iteration.}
      \step[\imps] $g^k(d_I)\notin D_k$
      \step[\imps] During the $k^{\text{th}}$ iteration
      \begin{align*}
        D & \gets D\cup\{d\}\\
        d & \gets g(d)
      \end{align*}
      \step[\iffs]
      \begin{alignat*}{3}
        D & \gets D_k\cup\{g^k(d_I)\} && = D_{k+1}
        \mathnote{Corollary~\ref{cor:g-seq}}
        \\ d & \gets g(g^k(d_I)) && = g^{k+1}(d_I)
        \tag*{\qedhere}
      \end{alignat*}
    \end{itemize}
  \end{proof}
\end{prop}

\begin{prop}[Termination]
  Algorithm~\ref{alg:fix-g} computes the least fixed-point of $G$.
  \begin{proof}
    Let $k_0\in\mathbb{N}$ be the smallest number such that $D_{k_0} = D_{k_0+1}$.
    \begin{itemize}
      \step[\imps] $g^{k_0}(d_I)\in D_{k_0}$
      \marginnote{Corollary~\ref{cor:g-seq}}
      \step[\imps] At iteration $k_0$ the loop condition of Algorithm~\ref{alg:fix-g} will fail
      \marginnote{By the correctness of maintenance.}
      \step[\imps] Algorithm~\ref{alg:fix-g} will return $D_{k_0}$\qedhere
    \end{itemize}
  \end{proof}
\end{prop}

\begin{lemma}\label{lemma:g-seq}
  $D_i = \{g^0(d_I),g^1(d_I),\ldots,g^{i-1}(d_I)\}$\\[1em]
  \noindent
  Proof by induction on $i$
  \begin{proof}[Base Case $i = 0$]
    $D_0 = \varnothing$\qedhere
  \end{proof}
  \begin{proof}[Inductive step $i = k + 1$]
    \begin{align*}
      D_{k+1} & = G(D_k)
      \mathnote{Definition of $D_i$}
      \\ & = G(\{d_I, g(d_I), g^2(d_I),\ldots,g^{k-1}(d_I)\})
      \mathnote{Induction Hypothesis}
      \\ & = \{d_I\}\cup\{g(d_I),g^2(d_I),g^3(d_I),\ldots,g^k(d_I)\}
      \mathnote{definition of $G$}
    \end{align*}\qedhere
  \end{proof}
\end{lemma}

\begin{corollary}\label{cor:g-seq}
  $D_{k+1} = D_k\cup\{g^k(d_I)\}$ follows by rearranging Lemma~\ref{lemma:g-seq}'s conclusion.
\end{corollary}

\begin{corollary}\label{cor:g-abs}
  If $(D_k)_{k\geq0}$ does not terminate in the first $k$ steps, then $\abs{D_k} = k$.
  \noindent
  Proof by Induction on $k$
  \begin{proof}[Base Case $k = 0$]
    $\abs{D_0} = 0$\qedhere
  \end{proof}
  \begin{proof}[Inductive Step $k = i + 1$]
    Suppose $(D_k)_{k\geq0}$ does not terminate in the first $i+1$ steps. Then it follows that:
    \begin{enumerate}[(i)]
      \item $D_{i+1}\neq D_i$
      \item $(D_k)_{k\geq0}$ did not terminate in the first $i$ steps.
    \end{enumerate}
    From which we can infer:
    \begin{align*}
      \abs{D_{i+1}} & = \abs{D_i\cup\{g^i(d_I)\}}
      \mathnote{Corollary~\ref{cor:g-seq}}
      \\ & = \abs{D_i} + 1
      \mathnote{By (i), $g^i(d_I)\notin D_i$}
      \\ & = i + 1
      \mathnote{By (ii), and the induction hypothesis}
    \end{align*}\qedhere
  \end{proof}
\end{corollary}


\section{Question 3}\label{sec:q-3}
\subsection{Part (a)}\label{sec:q-3-a}

Let $p_i : \mathit{Dist}\to[0,1]$ be defined as

\begin{align*}
  p_i(d) & = \sum_{\substack{s\in\mathit{St}\\s(z_i) = \mathit{true}}}d(s) && \text{for }1\leq i\leq n
\end{align*}

Then we may define our Galois Connection $\alpha : P(\mathit{Dist})\rightleftarrows \mathit{RA} : \gamma$ by:

\begin{align*}
  \alpha(D) & = \lambda z_i\ldotp~\alpha_0(\{p_i(d) : d\in D\}) \\
  \gamma(a) & = \{d\in\mathit{Dist}:\forall z_i\in\mathit{Var}\ldotp~p_i(d)\in\gamma_0(a(z_i))\}
\end{align*}

\begin{prop}
  $\alpha$, $\gamma$ forms a Galois Connection.\\
  It suffices to show that
  \begin{enumerate}[(i)]
    \item $\alpha$ is monotone
    \item $\gamma$ is monotone
    \item $\forall a\in\mathit{RA},D\in\mathit{Dist}\ldotp~\alpha(D)\sqsubseteq a\iff D\subseteq\gamma(a)$
  \end{enumerate}

  \begin{proof}[Proof (i)]
    let $D\subseteq D^\prime\subseteq\mathit{Dist}$
    \begin{itemize}
      \step[\imps] $\forall z_i\in\mathit{Var}\ldotp~\{p_i(d) : d\in D\} \subseteq \{p_i(d^\prime) : d^\prime\in D^\prime\}$
      \step[\imps] $\forall z_i\in\mathit{Var}\ldotp~\alpha_0(\{p_i(d) : d\in D\}) \sqsubseteq \alpha_0(\{p_i(d^\prime) : d^\prime\in D^\prime\})$
      \marginnote{$\alpha_0$ is monotonic}
      \step[\iffs] $\forall z_i\in\mathit{Var}\ldotp~\alpha(D)(z_i) \sqsubseteq\alpha(D^\prime)(z_i)$
      \marginnote{Definition of $\alpha$}
      \step[\iffs] $\alpha(D)\sqsubseteq\alpha(D^\prime)$\qedhere
      \marginnote{Definition of $\sqsubseteq_{\mathit{RA}}$}
    \end{itemize}
  \end{proof}

  \begin{proof}[Proof (ii)]
    Let $a,a^\prime\in\mathit{RA}$
    \begin{itemize}
      \step Suppose $a\sqsubseteq a^\prime$
      \step[\iffs] $\forall z_i\in\mathit{Var}\ldotp~a(z_i)\sqsubseteq a^\prime(z_i)$
      \marginnote{Definition of $\sqsubseteq_{\mathit{RA}}$}
      \step[\imps] $\forall z_i\in\mathit{Var}\ldotp~\gamma_0(a(z_i))\subseteq\gamma_0(a^\prime(z_i))$\hfill($\star$)
      \marginnote{$\gamma_0$ is monotonic}
      \step Let $d\in\gamma(a)$
      \step[\iffs] $\forall z_i\in\mathit{Var}\ldotp~p_i(d)\in\gamma_0(a(z_i))$
      \marginnote{Definition of $\gamma$}
      \step[\imps] $\forall z_i\in\mathit{Var}\ldotp~p_i(d)\in\gamma_0(a^\prime(z_i))$
      \marginnote{by $\star$}
      \step[\imps] $d\in\gamma(a^\prime)$
      \marginnote{Definition of $\gamma$}
      \step[\imps] $\gamma(a)\subseteq\gamma(a^\prime)$\qedhere
    \end{itemize}
  \end{proof}

  \begin{proof}[Proof (iii)]
    Let $a\in\mathit{RA},D\subseteq\mathit{Dist}$ and suppose $\alpha(D)\sqsubseteq a$
    \begin{itemize}
      \step[\iffs] $\forall z_i\in\mathit{Var}\ldotp~\alpha(D)(z_i)\sqsubseteq a(z_i)$
      \marginnote{Definition of $\sqsubseteq_{\mathit{RA}}$}
      \step[\iffs] $\forall z_i\in\mathit{Var}\ldotp~\alpha_0(\{p_i(d) : d\in D\})\sqsubseteq a(z_i)$
      \marginnote{Definition of $\alpha$}
      \step[\iffs] $\forall z_i\in\mathit{Var}\ldotp~\{p_i(d) : d\in D\}\subseteq \gamma_0(a(z_i))$
      \marginnote{$(\alpha_0,\gamma_0)$ is a Galois Connection}
      \step[\iffs] $\forall z_i\in\mathit{Var}\ldotp\forall d\in D\ldotp~p_i(d)\in\gamma_0(a(z_i))$
      \marginnote{Definition of $\subseteq$}
      \step[\iffs] $\forall d\in D\ldotp\forall z_i\in\mathit{Var}\ldotp~p_i(d)\in\gamma_0(a(z_i))$
      \marginnote{Swap $\forall$}
      \step[\iffs] $\forall d\in D\ldotp~d\in\{d\in\mathit{Dist} : \forall z_i\in\mathit{Var}\ldotp~p_i(d)\in\gamma_0(a(z_i))\}$
      \step[\iffs] $D\subseteq\{d\in\mathit{Dist} : \forall z_i\in\mathit{Var}\ldotp~p_i(d)\in\gamma_0(a(z_i))\}$
      \marginnote{Definition of $\subseteq$}
      \step[\iffs] $D\subseteq\gamma(a)$\qedhere
      \marginnote{Definition of $\gamma$}
    \end{itemize}
  \end{proof}
\end{prop}

\subsection{Part (b)}\label{sec:q-3-b}
\begin{algorithm}[htbp]
  \SetKwInOut{Input}{input}\SetKwInOut{Output}{output}

  \Input{$a\in\mathit{RA} = (\mathit{Var}\to\mathit{Range})$, a mapping in the abstract domain.}
  \Output{$\alpha(\gamma(a))$}
  \Begin{
      \For{$z_i\in\mathit{Var}$}{
        \If{$a(z_i) = \bot$} {
          return $\lambda z\ldotp\bot$\;
        }
      }
      return $a$\;
    }
  \caption{Closure in $\mathit{RA}$}\label{alg:close-ra}
\end{algorithm}

Consider Algorithm~\ref{alg:close-ra}, in order to prove its correctness, it suffices to equate the function it calculates with $\alpha\compose\gamma$, that is to say:
\begin{prop}
  \begin{align*}
    \alpha(\gamma(a)) & =
    \begin{cases}
      \lambda z_i\ldotp\bot & \exists z_j\in\mathit{Var}\ldotp~a(z_j) = \bot\\
      a & \text{otherwise}
    \end{cases}
  \end{align*}

  \begin{proof}
    Let $a\in\mathit{RA}$
    \begin{itemize}
      \step $\alpha(\gamma(a))$
      \step[=] $\alpha(\{d\in\mathit{Dist} : \forall z_j\in\mathit{Var}\ldotp~p_j(d)\in\gamma_0(a(z_j))\})$
      \marginnote{Definition of $\gamma$}
      \step[=] $\lambda z_i\ldotp\alpha_0(P_i)$ where
      \marginnote{Definition of $\alpha$}
      \begin{itemize}
        \step $P_i = \{p_i(d) : d\in\mathit{Dist},\forall z_j\in\mathit{Var}\ldotp~p_j(d)\in\gamma_0(a(z_j))\}$
      \end{itemize}
    \end{itemize}
    \begin{proof}[Case 1, $\exists z_k\in\mathit{Var}\ldotp~a(z_k)=\bot$]
      \begin{itemize}
        \step
        \begin{itemize}
          \step
          \begin{flalign*}
            \makebox[.5em]{}P_i & \subseteq \{p_i(d):d\in\mathit{Dist},~p_k(d)\in\gamma_0(a(z_k))\} &&
            \\ & = \{p_i(d):d\in\mathit{Dist},~p_k(d)\in\gamma_0(\bot)\} &&
            \\ & = \{p_i(d):d\in\mathit{Dist},~p_k(d)\in\varnothing\} &&
            \\ & = \varnothing
          \end{flalign*}
        \end{itemize}
        \step[\imps] $\alpha(\gamma(a)) = \lambda z_i\ldotp\alpha_0(\varnothing)$
        \step[\imps] $\alpha(\gamma(a)) = \lambda z_i\ldotp\bot$\qedhere
      \end{itemize}
    \end{proof}
    \begin{proof}[Case 2, {$\forall z_j\in\mathit{Var}\ldotp\exists l_j,u_j\in[0,1]\ldotp~l_j\leq u_j\wedge a(z_j) = [l_j,u_j]$}]
      \begin{itemize}
        \step
        \begin{itemize}
          \step
          \begin{flalign*}
           \makebox[.5em]{}P_i & = \{p_i(d):d\in\mathit{Dist},\forall z_j\in\mathit{Var}\ldotp~l_j\leq p_j(d)\leq u_j\}&&
          \end{flalign*}

        \end{itemize}
        \step Lemma~\ref{lem:dist-feasible} states that for any mapping of variables to probabilities, it is possible to find a distribution over states that achieves that mapping.
        \step[\imps] $\exists d\in\mathit{Dist}\ldotp\forall z_i\in\mathit{Var}\ldotp~p_i(d) = l_i$
        \itemsep-.1em
        \step[\wedge] $\exists d\in\mathit{Dist}\ldotp\forall z_i\in\mathit{Var}\ldotp~p_i(d) = u_i$
        \itemsep.5em
        \step[\imps] $\forall z_i\in\mathit{Var}\ldotp~\inf(P_i) = l_i \wedge\sup(P_i) = u_i$
        \step[\imps] $\alpha(\gamma(a)) = \lambda z_i\ldotp[l_i,u_i]$
        \step[\imps] $\alpha(\gamma(a)) = a$\qedhere
      \end{itemize}
    \end{proof}
  \end{proof}
\end{prop}

\begin{lemma}\label{lemma:dist-feasible}
  $\forall p : \mathit{Var}\to[0,1]\ldotp\exists d\in\mathit{Dist}\ldotp\forall z_i\in\mathit{Var}\ldotp~p(z_i) = p_i(d)$
  \begin{proof}
    Let $p : \mathit{Var}\to[0,1]$\\
    \noindent
    Consider the following linear system:
    \begin{align*}
      \sum_{\substack{s\in\mathit{St}\\s(z_i)=\mathit{true}}}d(s) & = p(z_i) &&\text{for } z_i\in\mathit{Var}\\
      \sum_{\substack{s\in\mathit{St}\\\phantom{s(z_i)=\mathit{true}}}} d(s) & = 1
    \end{align*}
    There are $2^n$ variables (in the system) --- corresponding to subsets of $\mathit{Var}$ --- and $n + 1$ constraints (one for each $z_i\in\mathit{Var}$, and an additional one to ensure that $d$ is a distribution)\\[1em]

    This can be interpreted in matrix form as $Sd = p$, where ${S\in\mathbb{B}^{(n+1)\times2^n}}$, ${d\in[0,1]^{2^n}}$, and ${p\in[0,1]^{n+1}}$. In this form, each row of $S$ is linearly independent, by their construction from the above constraints. Furthermore, as $\forall n\geq 0\ldotp~2^n\geq n+1$ there are always at least as many variables as constraints.\\[1em]

    \noindent
    $\implies$ $S$ has full rank.\\[1em]

    \noindent
    $\implies$ By Rouch\'e-Capelli theorem, the linear system is consistent.\\[1em]

    \noindent
    $\implies \exists d\in\mathit{Dist}\ldotp\forall z_i\in\mathit{Var}\ldotp~p(z_i) = p_i(d)$\qedhere
  \end{proof}
\end{lemma}

\subsection{Part (c)}\label{sec:q-3-c}


\end{document}
