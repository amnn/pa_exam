\subsection{Part (a)}\label{sec:q-2-a}
Define $\alpha : P(C)\to P(D)$ and $\gamma : P(D)\to P(C)$ by:
\begin{align*}
  \alpha(C_0) & = \{\beta(c) : c\in C_0\} \\
  \gamma(D_0) & = \{c\in C : \beta(c)\in D_0\}
\end{align*}

\begin{prop}
  $\alpha : P(C)\rightleftarrows P(D) : \gamma$ forms a Galois Connection\\
  It suffices to show that
  \begin{enumerate}[(i)]
    \item $\alpha$ is monotone
    \item $\gamma$ is monotone
    \item $\forall C_0\subseteq C,D_0\subseteq D.~\alpha(C_0)\subseteq D\iff C_0\subseteq\gamma(D_0)$
  \end{enumerate}
  \begin{proof}[Proof (i)]
    Let $C_0\subseteq C_0^\prime\subseteq C$
    \begin{itemize}
      \step Suppose $d\in\alpha(C_0)$
      \step[\imps] $d = \beta(c)$ for some $c\in C_0$
      \marginnote{Definition of $\alpha$}
      \step[\imps] $d = \beta(c)$ for some $c\in C_0^\prime$
      \marginnote{$C_0\subseteq C_0^\prime$}
      \step[\imps] $d\in\alpha(C_0^\prime)$
      \marginnote{Definition of $\alpha$}
      \step[\imps] $\alpha(C_0)\subseteq\alpha(C_0^\prime)$\qedhere
    \end{itemize}
  \end{proof}
  \begin{proof}[Proof (ii)]
    Let $D_0\subseteq D_0^\prime\subseteq D$
    \begin{itemize}
      \step Suppose $c\in\gamma(D_0)$
      \step[\imps] $\beta(c) \in D_0$
      \marginnote{Definition of $\gamma$}
      \step[\imps] $\beta(c) \in D_0^\prime$
      \marginnote{$D_0\subseteq D_0^\prime$}
      \step[\imps] $c\in\gamma(D_0^\prime)$
      \marginnote{Definition of $\gamma$}
      \step[\imps] $\gamma(D_0)\subseteq\gamma(D_0^\prime)$\qedhere
    \end{itemize}
  \end{proof}
  \begin{proof}[Proof (iii) $\Rightarrow$]
    Let $C_0\subseteq C, D_0\subseteq D$ such that $\alpha(C_0)\subseteq D_0$
    \begin{itemize}
      \step[\imps] $\{\beta(c) : c\in C_0\}\subseteq D_0$
      \marginnote{Definition of $\alpha$}
      \step[\imps] $\forall c\in C_0.~\beta(c)\in D_0$\hfill($\star$)
      \step let $c\in C_0$
      \step[\imps] $\beta(c)\in D_0$
      \marginnote{by $\star$}
      \step[\imps] $c\in\gamma(D_0)$
      \marginnote{Definition of $\gamma$}
      \step[\imps]$C_0\subseteq\gamma(D_0)$\qedhere
    \end{itemize}
  \end{proof}
  \begin{proof}[Proof (iii) $\Leftarrow$]
    Let $C_0\subseteq C, D_0\subseteq D$ such that $C_0\subseteq\gamma(D_0)$
    \begin{itemize}
      \step[\imps] $C_0\subseteq\{c\in C : \beta(c)\in D_0\}$
      \marginnote{Definition of $\gamma$}
      \step[\imps] $\forall c\in C_0.~\beta(c)\in D_0$\hfill($\star$)
      \step let $d\in \alpha(C_0)$
      \step[\imps] $d = \beta(c)$ for some $c\in C_0$
      \marginnote{Definition of $\alpha$}
      \step[\imps] $d\in D_0$
      \marginnote{by $\star$}
      \step[\imps]$\alpha(C_0)\subseteq D_0$\qedhere
    \end{itemize}
  \end{proof}
\end{prop}

\subsection{Part (b)}\label{sec:q-2-b}
Define $G : P(D)\to P(D)$
\begin{align*}
  G(D_0) = \{d_I\}\cup\{g(d) : d\in D_0\}
\end{align*}
\begin{prop}
  $\beta$ surjective $\implies G = \alpha\compose F\compose\gamma$
  \begin{proof}
    let $D_0\subseteq D$
    \begin{itemize}
      \step $\alpha(F(\gamma(D_0)))$
      \step[=] $\{\beta(c):c\in F(\gamma(D_0))\}$
      \marginnote{Definition of $\alpha$}
      \step[=] $\{\beta(c):c\in \{c_I\}\cup\{f(c):c\in\gamma(D_0)\}\}$
      \marginnote{Definition of $F$}
      \step[=] $\{\beta(c):c\in \{c_I\}\cup\{f(c):c\in\{c\in C:\beta(c)\in D_0\}\}\}$
      \marginnote{Definition of $\gamma$}
      \step[=] $\{\beta(c_I)\}\cup\{\beta(f(c)) : c\in C,\beta(c)\in D_0\}$
      \step[=] $\{d_I\}\cup\{g(\beta(c)) : c\in C,\beta(c)\in D_0\}$
      \marginnote{$\beta(c_I) = d_I$ and $\beta\compose f = g\compose\beta$}
      \step[=] $\{d_I\}\cup\{g(d):d\in D_0\}$
      \marginnote{$\beta$ surjective, by assumption}
      \step[=] $G(D_0)$
      \step[\imps] $G = \alpha\compose F\compose\gamma$\qedhere
    \end{itemize}
  \end{proof}
\end{prop}

\subsection{Part (c)}\label{sec:q-2-c}

Let $(D_k)_{k\geq0}$ be the sequence whose fixed-point is computed by our algorithm(s), defined by:
\begin{align*}
  D_0 & =\varnothing\\
  D_{k+1} & = G(D_k) && \text{for }k\geq0
\end{align*}

In order to calculate the worst-case time complexity, we note that Corollary~\ref{cor:g-abs} (given below) implies that if $(D_k)_{k\geq0}$ has not terminated after $\abs{D} + 1$ steps, $\abs{D_{\abs{D}+1}} = \abs{D} + 1$, which is a contradiction as $D$ is finite and $D_{\abs{D}+1}\subseteq D\implies \abs{D_{\abs{D}+1}}\leq \abs{D}$.\\[1em]

\noindent
$\implies (D_k)_{k\geq0}$ terminates in at most $\abs{D}+1$ steps.\\[1em]

As given in the question, every iteration of the algorithm takes, in the worst-case $O(\abs{D})$ time to run, so overall, this algorithm has a worst-case time complexity of $O(\abs{D}^2)$.\\[1em]

Taking inspiration from Corollary~\ref{cor:g-seq}, we may formulate an alternative algorithm as follows:

\begin{algorithm}
  \caption{An improved algorithm for calculating the fixed-point of $G$.}\label{alg:fix-g}
  \Begin{
      $D\gets\varnothing$\;
      $d\gets d_I$\;
      \While {$d\notin D$} {
        $D\gets D\cup\{d\}$\;
        $d\gets g(d)$
      }
      \KwRet $D$\;
    }
\end{algorithm}

As before, the outer-loop will take $\abs{D}+1$ iterations to terminate in the worst-case, but, the loop body now involves only a membership query, and an insertion, each taking $O(\log\abs{D})$ time in the worst-case, followed by the application of $g$ which we assume to be of unit cost.\\[1em]

\noindent
$\implies$ The worst-case time complexity of Algorithm~\ref{alg:fix-g} is $O(\abs{D}\log\abs{D})$.

\begin{prop}[Maintenance]
  At the beginning of iteration $i$, ${d = g^i(d_I)}$ and ${D = D_i}$.\\[1em]

  \noindent
  Proof by Induction
  \begin{proof}[Base Case $i = 0$]
    \begin{align*}
      D &= \varnothing = D_0\\
      d &= d_I = g^0(d_I)\tag*{\qedhere}
    \end{align*}
  \end{proof}
  \begin{proof}[Inductive Step $i = k + 1$]~\\
    \begin{itemize}
      \step At the beginning of iteration $k$, $D = D_k$, $d = g^k(d_I)$
      \marginnote{Induction Hypothesis}
      \step $d\notin D$ At the beginning of iteration $k$
      \marginnote{By assumption, we have reached the beginning of the $k+1^{\text{th}}$ iteration.}
      \step[\imps] $g^k(d_I)\notin D_k$
      \step[\imps] During the $k^{\text{th}}$ iteration
      \begin{align*}
        D & \gets D\cup\{d\}\\
        d & \gets g(d)
      \end{align*}
      \step[\iffs]
      \begin{alignat*}{3}
        D & \gets D_k\cup\{g^k(d_I)\} && = D_{k+1}
        \mathnote{Corollary~\ref{cor:g-seq}}
        \\ d & \gets g(g^k(d_I)) && = g^{k+1}(d_I)
        \tag*{\qedhere}
      \end{alignat*}
    \end{itemize}
  \end{proof}
\end{prop}

\begin{prop}[Termination]
  Algorithm~\ref{alg:fix-g} computes the least fixed-point of $G$.
  \begin{proof}
    Let $k_0\in\mathbb{N}$ be the smallest number such that $D_{k_0} = D_{k_0+1}$.
    \begin{itemize}
      \step[\imps] $g^{k_0}(d_I)\in D_{k_0}$
      \marginnote{Corollary~\ref{cor:g-seq}}
      \step[\imps] At iteration $k_0$ the loop condition of Algorithm~\ref{alg:fix-g} will fail
      \marginnote{By the correctness of maintenance.}
      \step[\imps] Algorithm~\ref{alg:fix-g} will return $D_{k_0}$\qedhere
    \end{itemize}
  \end{proof}
\end{prop}

\begin{lemma}\label{lemma:g-seq}
  $D_i = \{g^0(d_I),g^1(d_I),\ldots,g^{i-1}(d_I)\}$\\[1em]
  \noindent
  Proof by induction on $i$
  \begin{proof}[Base Case $i = 0$]
    $D_0 = \varnothing$\qedhere
  \end{proof}
  \begin{proof}[Inductive step $i = k + 1$]
    \begin{align*}
      D_{k+1} & = G(D_k)
      \mathnote{Definition of $D_i$}
      \\ & = G(\{d_I, g(d_I), g^2(d_I),\ldots,g^{k-1}(d_I)\})
      \mathnote{Induction Hypothesis}
      \\ & = \{d_I\}\cup\{g(d_I),g^2(d_I),g^3(d_I),\ldots,g^k(d_I)\}
      \mathnote{definition of $G$}
    \end{align*}\qedhere
  \end{proof}
\end{lemma}

\begin{corollary}\label{cor:g-seq}
  $D_{k+1} = D_k\cup\{g^k(d_I)\}$ follows by rearranging Lemma~\ref{lemma:g-seq}'s conclusion.
\end{corollary}

\begin{corollary}\label{cor:g-abs}
  If $(D_k)_{k\geq0}$ does not terminate in the first $k$ steps, then $\abs{D_k} = k$.
  \noindent
  Proof by Induction on $k$
  \begin{proof}[Base Case $k = 0$]
    $\abs{D_0} = 0$\qedhere
  \end{proof}
  \begin{proof}[Inductive Step $k = i + 1$]
    Suppose $(D_k)_{k\geq0}$ does not terminate in the first $i+1$ steps. Then it follows that:
    \begin{enumerate}[(i)]
      \item $D_{i+1}\neq D_i$
      \item $(D_k)_{k\geq0}$ did not terminate in the first $i$ steps.
    \end{enumerate}
    From which we can infer:
    \begin{align*}
      \abs{D_{i+1}} & = \abs{D_i\cup\{g^i(d_I)\}}
      \mathnote{Corollary~\ref{cor:g-seq}}
      \\ & = \abs{D_i} + 1
      \mathnote{By (i), $g^i(d_I)\notin D_i$}
      \\ & = i + 1
      \mathnote{By (ii), and the induction hypothesis}
    \end{align*}\qedhere
  \end{proof}
\end{corollary}
